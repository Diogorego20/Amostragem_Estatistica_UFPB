\documentclass[12pt,a4paper]{article}
\usepackage[brazil]{babel}
\usepackage[utf8]{inputenc}
\usepackage[T1]{fontenc}
\usepackage{geometry}
\usepackage{xcolor}
\usepackage{listings}
\usepackage{amsmath}
\usepackage{hyperref}
\usepackage{booktabs}
\usepackage{caption}
\usepackage{graphicx}

\geometry{a4paper, margin=2.5cm}

\definecolor{codegreen}{rgb}{0,0.6,0}
\definecolor{codegray}{rgb}{0.5,0.5,0.5}
\definecolor{codepurple}{rgb}{0.58,0,0.82}
\definecolor{backcolour}{rgb}{0.95,0.95,0.92}

\lstdefinestyle{r-style}{
    backgroundcolor=\color{backcolour},   
    commentstyle=\color{codegreen},
    keywordstyle=\color{blue},
    numberstyle=\tiny\color{codegray},
    stringstyle=\color{codepurple},
    basicstyle=\ttfamily\footnotesize,
    breakatwhitespace=false,         
    breaklines=true,                 
    captionpos=b,                    
    keepspaces=true,                 
    numbers=left,                    
    numbersep=8pt,                  
    showspaces=false,                
    showstringspaces=false,
    showtabs=false,                  
    tabsize=2,
    frame=single,
    language=R
}

\lstdefinestyle{python-style}{
    backgroundcolor=\color{backcolour},   
    commentstyle=\color{codegreen},
    keywordstyle=\color{magenta},
    numberstyle=\tiny\color{codegray},
    stringstyle=\color{codepurple},
    basicstyle=\ttfamily\footnotesize,
    breakatwhitespace=false,         
    breaklines=true,                 
    captionpos=b,                    
    keepspaces=true,                 
    numbers=left,                    
    numbersep=8pt,                  
    showspaces=false,                
    showstringspaces=false,
    showtabs=false,                  
    tabsize=2,
    frame=single,
    language=Python
}

\title{Análise de Amostragem Estatística \\ \large Implementações em R e Python}
\author{Diogo Rego}
\date{\today}

\begin{document}

\maketitle

\begin{abstract}
Este documento apresenta implementações de análise de amostragem estatística, incorporando boas práticas de programação. Desenvolvido como material didático, aborda conceitos fundamentais de inferência estatística através de exemplos práticos em R e Python, com foco em validação, tratamento de erros e eficiência computacional.
\end{abstract}

\tableofcontents

\section{Introdução à Amostragem Estatística}
\label{sec:introducao}

A amostragem estatística é um pilar fundamental da inferência estatística, permitindo que conclusões sobre uma população sejam obtidas a partir do estudo de um subconjunto representativo. Este material aborda os seguintes conceitos teóricos:

\subsection{Conceitos Fundamentais}
\begin{itemize}
\item \textbf{População}: Conjunto completo de elementos sob estudo
\item \textbf{Amostra}: Subconjunto representativo da população
\item \textbf{Parâmetro}: Medida descritiva da população
\item \textbf{Estatística}: Medida descritiva da amostra
\item \textbf{Distribuição Amostral}: Distribuição de uma estatística sobre todas as amostras possíveis
\end{itemize}

\subsection{Contexto do Exercício}
O exercício proposto considera uma população finita \( U = \{1,2,3,4,5,6,7,8\} \) com variável de interesse \( Y = \{1,2,4,4,7,7,7,8\} \). O objetivo é analisar todas as amostras possíveis de tamanho \( n = 4 \) e estudar o comportamento das estatísticas amostrais.

A tabela abaixo resume a evolução das implementações, destacando as melhorias nas boas práticas de programação:

\begin{table}[h]
\centering
\begin{tabular}{lccc}
\toprule
\textbf{Boa Prática} & \textbf{Versão Original} & \textbf{Versão Corrigida} & \textbf{Status} \\
\midrule
Validação de Entrada & ❌ & ✅ & Implementado \\
Tratamento de Erros & ❌ & ✅ & Implementado \\
Eficiência Computacional & ⚠️ & ✅ & Otimizado \\
Modularização & ❌ & ✅ & Modularizado \\
Documentação & ✅ & ✅ & Mantido \\
Nomenclatura & ✅ & ✅ & Mantido \\
\bottomrule
\end{tabular}
\caption{Evolução das boas práticas implementadas}
\end{table}

\section{Fundamentos Teóricos da Amostragem}
\label{sec:fundamentos-teoricos}

\subsection{Combinações Possíveis}
Para uma população de tamanho \( N \) e amostras de tamanho \( n \), o número total de combinações possíveis sem reposição é dado por:

\[
\text{Combinações} = \binom{N}{n} = \frac{N!}{n!(N-n)!}
\]

No exercício em questão:
\[
\binom{8}{4} = \frac{8!}{4!4!} = 70 \text{ combinações possíveis}
\]

\subsection{Estatísticas de Interesse}
Para cada amostra, calculamos:
\begin{itemize}
\item \textbf{Média amostral}: \( \bar{x} = \frac{\sum_{i=1}^{n} x_i}{n} \)
\item \textbf{Proporção}: \( p = \frac{\text{número de elementos > limite}}{n} \)
\item \textbf{Variância amostral}: \( s^2 = \frac{\sum_{i=1}^{n} (x_i - \bar{x})^2}{n-1} \)
\end{itemize}

\subsection{Importância da Validação}
A validação de entradas é crucial em aplicações estatísticas para garantir:
\begin{itemize}
\item Consistência dos parâmetros (\( n \leq N \))
\item Integridade dos dados (ausência de NA/NaN)
\item Viabilidade computacional
\end{itemize}

\section{Implementação R Completa}
\label{sec:r-completa}

\subsection{Abordagem Didática}
A implementação em R segue uma estrutura modular que facilita o entendimento dos conceitos de amostragem. Cada função corresponde a uma etapa do processo inferencial.

\begin{lstlisting}[style=r-style, caption={Implementação R para análise de amostragem}]
#' =============================================================================
#' ANÁLISE DE AMOSTRAGEM ESTATÍSTICA - MATERIAL DIDÁTICO
#' 
#' Objetivos de aprendizagem:
#' 1. Compreender distribuições amostrais
#' 2. Implementar validação estatística
#' 3. Desenvolver algoritmos eficientes para amostragem
#' =============================================================================

# PACOTES NECESSÁRIOS ----------------------------------------------------------
if (!require(combinat)) install.packages("combinat")
library(combinat)

# VALIDAÇÃO DE ENTRADAS --------------------------------------------------------

#' @title Validar parâmetros de entrada
#' @description Implementa verificações fundamentais para análise estatística
#' @param valores Vetor numérico com valores da população
#' @param n Tamanho da amostra (deve ser <= length(valores))
#' @param limite Valor limite para cálculo de proporções
#' @return Lista com status e mensagens de erro
validar_entradas <- function(valores, n, limite) {
  erros <- character()
  avisos <- character()
  
  # Validações críticas baseadas em teoria de amostragem
  if (!is.numeric(valores)) {
    erros <- c(erros, "Valores da população devem ser numéricos")
  }
  
  if (any(is.na(valores))) {
    erros <- c(erros, "Valores da população não podem conter NA")
  }
  
  if (!is.numeric(n) || n < 1) {
    erros <- c(erros, "Tamanho da amostra deve ser número positivo")
  }
  
  # Verificação fundamental: n não pode ser maior que N
  if (n > length(valores)) {
    erros <- c(erros, sprintf(
      "Tamanho da amostra (n=%d) maior que população (N=%d)", 
      n, length(valores)
    ))
  }
  
  # Validações de alerta para boas práticas estatísticas
  if (length(valores) < 5) {
    avisos <- c(avisos, "População muito pequena para inferência")
  }
  
  if (n < 30) {
    avisos <- c(avisos, "Amostra pequena - cuidado com estimativas")
  }
  
  return(list(
    sucesso = length(erros) == 0,
    erros = erros,
    avisos = avisos
  ))
}

# GERADOR DE AMOSTRAS INTELIGENTE ----------------------------------------------

#' @title Gerar amostras de forma eficiente
#' @description Demonstra diferentes estratégias de amostragem baseadas no contexto
#' @param valores Vetor com valores da população
#' @param n Tamanho da amostra
#' @param max_combinacoes Número máximo de combinações para cálculo exaustivo
#' @return Lista contendo amostras e metadados
gerar_amostras_inteligente <- function(valores, n, max_combinacoes = 10000) {
  total_combinacoes <- choose(length(valores), n)
  
  cat(sprintf("Combinações possíveis: %d\n", total_combinacoes))
  
  if (total_combinacoes <= max_combinacoes && total_combinacoes > 0) {
    # Estratégia exaustiva para combinações viáveis
    cat("Usando método: EXAUSTIVO (todas combinações)\n")
    amostras <- combinat::combn(valores, n)
    metodo <- "exaustivo"
    amostras_geradas <- total_combinacoes
  } else {
    # Estratégia de amostragem para muitas combinações (abordagem prática)
    amostras_geradas <- min(max_combinacoes, 5000)
    cat(sprintf("Usando método: AMOSTRAGEM (%d amostras)\n", amostras_geradas))
    
    amostras <- replicate(
      amostras_geradas, 
      sample(valores, n, replace = FALSE), 
      simplify = FALSE
    )
    amostras <- do.call(cbind, amostras)
    metodo <- "amostragem"
  }
  
  return(list(
    amostras = amostras,
    metodo = metodo,
    total_combinacoes = total_combinacoes,
    amostras_geradas = ncol(amostras)
  ))
}

# CÁLCULO DE ESTATÍSTICAS ------------------------------------------------------

#' @title Calcular estatísticas descritivas
#' @description Calcula medidas de tendência central e dispersão para amostras
#' @param amostras Matrix onde cada coluna é uma amostra
#' @param limite Valor limite para cálculo de proporção
#' @return Dataframe com estatísticas calculadas
calcular_estatisticas <- function(amostras, limite = 4) {
  tryCatch({
    # Medidas de tendência central
    medias <- apply(amostras, 2, mean)
    proporcoes <- apply(amostras, 2, function(x) mean(x > limite))
    
    # Medidas de dispersão
    desvios_padrao <- apply(amostras, 2, sd)
    
    data.frame(
      media = medias,
      proporcao = proporcoes,
      desvio_padrao = desvios_padrao
    )
  }, error = function(e) {
    stop("Erro no cálculo de estatísticas: ", e$message)
  })
}

# ANÁLISE EXPLORATÓRIA ---------------------------------------------------------

#' @title Gerar análise exploratória das estatísticas
#' @description Produz sumários descritivos para compreensão das distribuições
#' @param estatisticas Dataframe com estatísticas das amostras
gerar_analise_exploratoria <- function(estatisticas) {
  cat("\n=== ANÁLISE EXPLORATÓRIA DAS ESTATÍSTICAS ===\n")
  
  cat("\nESTATÍSTICAS DAS MÉDIAS:\n")
  cat(sprintf("Média das médias amostrais: %.3f\n", mean(estatisticas$media)))
  cat(sprintf("Variância das médias: %.3f\n", var(estatisticas$media)))
  cat(sprintf("Desvio padrão das médias: %.3f\n", sd(estatisticas$media)))
  
  cat("\nESTATÍSTICAS DAS PROPORÇÕES:\n")
  cat(sprintf("Média das proporções: %.3f\n", mean(estatisticas$proporcao)))
  cat(sprintf("Variância das proporções: %.3f\n", var(estatisticas$proporcao)))
  
  # Teorema do Limite Central - verificação prática
  cat("\nVERIFICAÇÃO DO TEOREMA DO LIMITE CENTRAL:\n")
  cat(sprintf("Erro padrão da média: %.3f\n", sd(estatisticas$media)))
}

# FUNÇÃO PRINCIPAL ORQUESTRADORA -----------------------------------------------

#' @title Executar análise completa de amostragem
#' @description Integra todas as etapas do processo de inferência estatística
#' @param valores Vetor com valores da população
#' @param n Tamanho da amostra
#' @param limite Valor limite para proporções (default=4)
#' @param max_combinacoes Limite para cálculo exaustivo (default=10000)
#' @return Lista completa com resultados e metadados
executar_analise_completa <- function(valores, n, limite = 4, max_combinacoes = 10000) {
  cat("=== INÍCIO DA ANÁLISE DE AMOSTRAGEM ===\n")
  
  resultado_final <- list(sucesso = FALSE)
  
  tryCatch({
    # 1. VALIDAÇÃO (Etapa fundamental)
    cat("1. Validando entradas...\n")
    validacao <- validar_entradas(valores, n, limite)
    
    if (!validacao$sucesso) {
      stop(paste(validacao$erros, collapse = "; "))
    }
    
    # Exibir avisos para conscientização estatística
    if (length(validacao$avisos) > 0) {
      cat("AVISOS ESTATÍSTICOS:\n")
      for (aviso in validacao$avisos) {
        cat(sprintf("  - %s\n", aviso))
      }
    }
    
    # 2. GERAÇÃO DE AMOSTRAS
    cat("2. Gerando amostras...\n")
    resultado_amostras <- gerar_amostras_inteligente(valores, n, max_combinacoes)
    
    # 3. CÁLCULO DE ESTATÍSTICAS
    cat("3. Calculando estatísticas...\n")
    estatisticas <- calcular_estatisticas(resultado_amostras$amostras, limite)
    
    # 4. ANÁLISE EXPLORATÓRIA
    cat("4. Realizando análise exploratória...\n")
    gerar_analise_exploratoria(estatisticas)
    
    # 5. MONTAGEM DO DATAFRAME FINAL
    cat("5. Consolidando resultados...\n")
    dados_amostras <- as.data.frame(t(resultado_amostras$amostras))
    colnames(dados_amostras) <- paste0("V", 1:ncol(resultado_amostras$amostras))
    
    resultado_final <- cbind(dados_amostras, estatisticas)
    
    # 6. METADADOS PARA DOCUMENTAÇÃO
    metadados <- list(
      metodo = resultado_amostras$metodo,
      total_combinacoes = resultado_amostras$total_combinacoes,
      amostras_geradas = resultado_amostras$amostras_geradas,
      populacao = valores,
      parametros = list(n = n, limite = limite),
      timestamp = Sys.time()
    )
    
    cat("=== ANÁLISE CONCLUÍDA COM SUCESSO ===\n")
    
    return(list(
      sucesso = TRUE,
      dados = resultado_final,
      metadados = metadados,
      validacao = validacao
    ))
    
  }, error = function(e) {
    cat(sprintf("=== ERRO NA ANÁLISE: %s ===\n", e$message))
    
    return(list(
      sucesso = FALSE,
      erro = e$message,
      timestamp = Sys.time()
    ))
  })
}

# EXECUÇÃO DIDÁTICA ------------------------------------------------------------

# Dados do exercício proposto
populacao <- c(1, 2, 4, 4, 7, 7, 7, 8)
tamanho_amostra <- 4
limite_proporcao <- 4

cat("EXECUTANDO ANÁLISE COM DADOS DO EXERCÍCIO:\n")
cat(sprintf("População: %s\n", paste(populacao, collapse = ", ")))
cat(sprintf("Tamanho da amostra: %d\n", tamanho_amostra))
cat(sprintf("Limite para proporção: %d\n", limite_proporcao))
cat(sprintf("Número esperado de combinações: %d\n", choose(8, 4)))
cat("\n")

# Executar análise completa
resultado <- executar_analise_completa(
  valores = populacao,
  n = tamanho_amostra,
  limite = limite_proporcao,
  max_combinacoes = 10000
)

# Exibir resultados para fins didáticos
if (resultado$sucesso) {
  cat(sprintf("\nRESUMO DOS RESULTADOS:\n"))
  cat(sprintf("- Método utilizado: %s\n", resultado$metadados$metodo))
  cat(sprintf("- Amostras analisadas: %d\n", nrow(resultado$dados)))
  
  cat("\nPRIMEIRAS 5 LINHAS DOS RESULTADOS:\n")
  print(head(resultado$dados, 5))
  
  # Análise adicional para compreensão conceitual
  cat("\nCOMPREENSÃO CONCEITUAL:\n")
  cat("A distribuição das médias amostrais aproxima-se de uma distribuição\n")
  cat("normal (Teorema do Limite Central), mesmo para amostras relativamente\n")
  cat("pequenas, quando a população não é muito assimétrica.\n")
  
} else {
  cat(sprintf("FALHA NA ANÁLISE: %s\n", resultado$erro))
  cat("Esta falha demonstra a importância da validação em análises estatísticas.\n")
}
\end{lstlisting}

\subsection{Aplicação Shiny para Ensino}
\label{subsec:shiny-corrigido}

\begin{lstlisting}[style=r-style, caption={Aplicação Shiny interativa para ensino de amostragem}]
# APLICAÇÃO SHINY PARA ENSINO DE AMOSTRAGEM ESTATÍSTICA
library(shiny)
library(combinat)

# UI ---------------------------------------------------------------------------
ui <- fluidPage(
  titlePanel("�� Laboratório de Amostragem Estatística"),
  sidebarLayout(
    sidebarPanel(
      h4("Configuração da População"),
      numericInput('N', 'Tamanho da população (N):', 8, min = 2, max = 50),
      textInput('valores', 'Valores da população (separados por vírgula):', 
                '1,2,4,4,7,7,7,8'),
      
      h4("Parâmetros da Amostra"),
      numericInput('n', 'Tamanho da amostra (n):', 4, min = 1, max = 20),
      numericInput('limite', 'Valor limite para proporção:', 4),
      
      h4("Configurações Avançadas"),
      sliderInput('max_combinacoes', 'Máximo de combinações para análise:', 
                  min = 100, max = 10000, value = 5000, step = 100),
      
      actionButton('gerar', '�� Executar Análise', class = "btn-primary"),
      
      # Painel educacional
      conditionalPanel(
        condition = "input.gerar > 0",
        wellPanel(
          h5("Conceitos Estatísticos:"),
          tags$ul(
            tags$li("Distribuição amostral da média"),
            tags$li("Teorema do Limite Central"), 
            tags$li("Variabilidade amostral"),
            tags$li("Estimação de parâmetros")
          )
        )
      )
    ),
    
    mainPanel(
      # Status e alertas
      verbatimTextOutput('status'),
      uiOutput('alertas'),
      
      # Resultados principais
      h3("Resultados da Análise Amostral"),
      tableOutput('tabela_resultados'),
      
      # Visualizações educacionais
      h4("Distribuições Amostrais"),
      fluidRow(
        column(6, 
               plotOutput('hist_medias'),
               p("Distribuição das Médias Amostrais - Ilustra o Teorema do Limite Central")
        ),
        column(6, 
               plotOutput('hist_proporcoes'),
               p("Distribuição das Proporções - Mostra a variabilidade das estimativas")
        )
      ),
      
      # Estatísticas descritivas
      h4("Estatísticas Descritivas"),
      verbatimTextOutput('estatisticas_descritivas'),
      
      # Download para relatórios
      downloadButton('download', '�� Exportar Resultados')
    )
  )
)

# SERVER -----------------------------------------------------------------------
server <- function(input, output, session) {
  
  # Reactive values para estado da aplicação
  valores_reactive <- reactiveVal()
  resultados_reactive <- reactiveVal()
  
  # Atualizar valores da população
  observe({
    req(input$valores)
    tryCatch({
      valores <- as.numeric(unlist(strsplit(input$valores, ',')))
      valores_reactive(valores)
    }, error = function(e) {
      showNotification("Erro ao processar valores da população", type = "error")
    })
  })
  
  # Executar análise quando o botão for pressionado
  observeEvent(input$gerar, {
    # Reiniciar outputs
    output$alertas <- renderUI({})
    output$estatisticas_descritivas <- renderText({""})
    
    tryCatch({
      # Obter e validar dados
      valores <- valores_reactive()
      n <- input$n
      limite <- input$limite
      
      # Executar análise com feedback visual
      withProgress(message = 'Processando análise...', value = 0, {
        incProgress(0.2, detail = "Validando parâmetros estatísticos")
        
        # Validar entradas
        validacao <- validar_entradas(valores, n, limite)
        if (!validacao$sucesso) {
          stop(paste(validacao$erros, collapse = "; "))
        }
        
        incProgress(0.4, detail = "Gerando amostras")
        
        # Gerar amostras
        resultado_amostras <- gerar_amostras_inteligente(valores, n, input$max_combinacoes)
        
        incProgress(0.6, detail = "Calculando estatísticas")
        
        # Calcular estatísticas
        estatisticas <- calcular_estatisticas(resultado_amostras$amostras, limite)
        
        # Preparar dados finais
        dados_amostras <- as.data.frame(t(resultado_amostras$amostras))
        colnames(dados_amostras) <- paste0("obs_", 1:ncol(resultado_amostras$amostras))
        resultado_final <- cbind(dados_amostras, estatisticas)
        
        # Armazenar resultados
        resultados_reactive(list(
          sucesso = TRUE,
          dados = resultado_final,
          metadados = list(
            metodo = resultado_amostras$metodo,
            total_combinacoes = resultado_amostras$total_combinacoes,
            amostras_geradas = resultado_amostras$amostras_geradas
          ),
          validacao = validacao
        ))
        
        incProgress(0.8, detail = "Gerando visualizações")
      })
      
      # Output de status educativo
      output$status <- renderText({
        resultado <- resultados_reactive()
        if (resultado$sucesso) {
          sprintf("✅ Análise concluída!\nMétodo: %s\nAmostras analisadas: %d\nCombinações possíveis: %d", 
                  resultado$metadados$metodo,
                  resultado$metadados$amostras_geradas,
                  resultado$metadados$total_combinacoes)
        }
      })
      
      # Estatísticas descritivas educativas
      output$estatisticas_descritivas <- renderText({
        resultado <- resultados_reactive()
        if (resultado$sucesso) {
          dados <- resultado$dados
          sprintf(
            "ESTATÍSTICAS DAS DISTRIBUIÇÕES AMOSTRAIS:\n\nMÉDIAS:\n- Média: %.3f\n- Desvio Padrão: %.3f\n- Coef. Variação: %.3f\n\nPROPORÇÕES:\n- Média: %.3f\n- Desvio Padrão: %.3f",
            mean(dados$media), sd(dados$media), sd(dados$media)/mean(dados$media),
            mean(dados$proporcao), sd(dados$proporcao)
          )
        }
      })
      
    }, error = function(e) {
      output$status <- renderText(sprintf("❌ Erro na análise: %s", e$message))
      resultados_reactive(NULL)
    })
  })
  
  # Tabela de resultados
  output$tabela_resultados <- renderTable({
    req(resultados_reactive())
    if (resultados_reactive()$sucesso) {
      head(resultados_reactive()$dados, 10)
    }
  }, striped = TRUE, hover = TRUE, caption = "Primeiras 10 amostras e suas estatísticas")
  
  # Histograma das médias com anotações educativas
  output$hist_medias <- renderPlot({
    req(resultados_reactive())
    if (resultados_reactive()$sucesso) {
      dados <- resultados_reactive()$dados
      hist(dados$media, 
           main = "Distribuição das Médias Amostrais",
           xlab = "Média Amostral", 
           ylab = "Frequência",
           col = "lightblue",
           breaks = 20)
      abline(v = mean(dados$media), col = "red", lwd = 2)
      legend("topright", legend = "Média das Médias", col = "red", lwd = 2)
    }
  })
  
  # Histograma das proporções
  output$hist_proporcoes <- renderPlot({
    req(resultados_reactive())
    if (resultados_reactive()$sucesso) {
      dados <- resultados_reactive()$dados
      hist(dados$proporcao, 
           main = "Distribuição das Proporções",
           xlab = "Proporção", 
           ylab = "Frequência",
           col = "lightgreen",
           breaks = 20)
      abline(v = mean(dados$proporcao), col = "red", lwd = 2)
      legend("topright", legend = "Média das Proporções", col = "red", lwd = 2)
    }
  })
  
  # Download dos resultados
  output$download <- downloadHandler(
    filename = function() {
      paste("resultados_amostragem_", format(Sys.time(), "%Y%m%d_%H%M%S"), ".csv", sep = "")
    },
    content = function(file) {
      req(resultados_reactive())
      if (resultados_reactive()$sucesso) {
        write.csv(resultados_reactive()$dados, file, row.names = FALSE)
      }
    }
  )
}

# EXECUTAR APLICAÇÃO -----------------------------------------------------------
shinyApp(ui, server)
\end{lstlisting}

\section{Implementação Python para Análise Estatística}
\label{sec:python}

\begin{lstlisting}[style=python-style, caption={Implementação Python educativa para amostragem}]
"""
===============================================================================
LABORATÓRIO DE AMOSTRAGEM ESTATÍSTICA - PYTHON
Material didático para ensino de inferência estatística

Conceitos abordados:
- Distribuições amostrais
- Teorema do Limite Central  
- Validação estatística
- Visualização de resultados
===============================================================================
"""

import streamlit as st
import pandas as pd
import numpy as np
import itertools
from math import comb
from datetime import datetime
import traceback

# CONFIGURAÇÃO DA PÁGINA ======================================================
st.set_page_config(
    page_title="Laboratório de Amostragem Estatística",
    page_icon="��",
    layout="wide",
    initial_sidebar_state="expanded"
)

# FUNÇÕES DE VALIDAÇÃO ESTATÍSTICA ============================================

def validar_entradas(valores: list, n: int, N: int, limite: float) -> dict:
    """
    Valida parâmetros estatísticos baseado em princípios de amostragem.
    
    Args:
        valores: Lista de valores da população
        n: Tamanho da amostra
        N: Tamanho da população
        limite: Valor limite para proporções
        
    Returns:
        Dict com status e mensagens de erro/aviso
    """
    erros = []
    avisos = []
    
    # Validações baseadas em teoria de amostragem
    if not all(isinstance(x, (int, float)) for x in valores):
        erros.append("Todos os valores devem ser numéricos")
    
    if any(pd.isna(x) for x in valores):
        erros.append("Valores não podem conter NaN")
    
    if not isinstance(n, int) or n < 1:
        erros.append("Tamanho da amostra deve ser inteiro positivo")
    
    # Princípio fundamental: n <= N
    if n > len(valores):
        erros.append(f"Tamanho da amostra (n={n}) maior que população (N={len(valores)})")
    
    if len(valores) != N:
        avisos.append(f"Número de valores ({len(valores)}) diferente do N informado ({N})")
    
    # Alertas para boas práticas estatísticas
    if len(valores) < 5:
        avisos.append("População muito pequena para inferência estatística")
    
    if n < 30:
        avisos.append("Amostra pequena - cuidado com estimativas")
    
    return {
        'sucesso': len(erros) == 0,
        'erros': erros,
        'avisos': avisos
    }

# GERADOR DE AMOSTRAS COM ABORDAGEM EDUCATIVA =================================

def gerar_amostras_inteligente(valores: list, n: int, max_combinacoes: int = 10000) -> dict:
    """
    Gera amostras usando diferentes estratégias baseadas no contexto educativo.
    
    Args:
        valores: Lista de valores da população
        n: Tamanho da amostra
        max_combinacoes: Limite para cálculo exaustivo
        
    Returns:
        Dict com amostras e metadados educativos
    """
    try:
        total_combinacoes = comb(len(valores), n)
        st.info(f"**Conceito:** Número de combinações possíveis: {total_combinacoes:,}")
        
        if total_combinacoes <= max_combinacoes and total_combinacoes > 0:
            # Abordagem exaustiva para compreensão completa
            st.success("**Método:** EXAUSTIVO - Analisando todas combinações possíveis")
            combinacoes = list(itertools.combinations(valores, n))
            metodo = "exaustivo"
            amostras_geradas = total_combinacoes
        else:
            # Abordagem prática para grandes populações
            amostras_geradas = min(max_combinacoes, 5000)
            st.warning(f"**Método:** AMOSTRAGEM - {amostras_geradas} amostras (abordagem prática)")
            
            combinacoes = []
            for _ in range(amostras_geradas):
                amostra = np.random.choice(valores, size=n, replace=False)
                combinacoes.append(tuple(amostra))
            metodo = "amostragem"
        
        return {
            'combinacoes': combinacoes,
            'metodo': metodo,
            'total_combinacoes': total_combinacoes,
            'amostras_geradas': len(combinacoes)
        }
        
    except Exception as e:
        st.error(f"Erro na geração de amostras: {str(e)}")
        raise

# CÁLCULO DE ESTATÍSTICAS DESCRITIVAS =========================================

def calcular_estatisticas(combinacoes: list, limite: float) -> pd.DataFrame:
    """
    Calcula estatísticas descritivas para análise das distribuições amostrais.
    
    Args:
        combinacoes: Lista de tuplas com as amostras
        limite: Valor limite para cálculo de proporção
        
    Returns:
        DataFrame com estatísticas calculadas
    """
    try:
        dados = []
        
        for i, amostra in enumerate(combinacoes):
            # Medidas de tendência central
            media = np.mean(amostra)
            proporcao = np.mean([x > limite for x in amostra])
            
            # Medidas de dispersão
            desvio_padrao = np.std(amostra, ddof=1)
            
            # Coletar dados para análise
            registro = {
                'amostra_id': i + 1,
                'media': media,
                'proporcao': proporcao,
                'desvio_padrao': desvio_padrao
            }
            
            # Incluir valores individuais para análise detalhada
            for j, valor in enumerate(amostra):
                registro[f'valor_{j+1}'] = valor
            
            dados.append(registro)
        
        return pd.DataFrame(dados)
        
    except Exception as e:
        st.error(f"Erro no cálculo de estatísticas: {str(e)}")
        raise

# ANÁLISE EXPLORATÓRIA EDUCATIVA ==============================================

def gerar_analise_exploratoria(dados: pd.DataFrame) -> None:
    """
    Gera análise exploratória com foco educativo.
    
    Args:
        dados: DataFrame com estatísticas das amostras
    """
    st.write("### �� Análise Exploratória das Distribuições Amostrais")
    
    col1, col2, col3 = st.columns(3)
    
    with col1:
        st.metric("Média das Médias", f"{dados['media'].mean():.3f}")
        st.metric("EP das Médias", f"{dados['media'].std():.3f}")
        
    with col2:
        st.metric("Média das Proporções", f"{dados['proporcao'].mean():.3f}")
        st.metric("EP das Proporções", f"{dados['proporcao'].std():.3f}")
        
    with col3:
        st.metric("Variabilidade Médias", f"{dados['media'].std()/dados['media'].mean():.3f}")
        st.metric("Amostras Analisadas", len(dados))
    
    # Insights educativos
    st.info("""
    **Insights Estatísticos:**
    - A distribuição das médias amostrais aproxima-se da normalidade (TLC)
    - O erro padrão mede a variabilidade das estimativas
    - Proporções seguem distribuição binomial
    """)

# FUNÇÃO PRINCIPAL DE ANÁLISE =================================================

def executar_analise_completa(valores: list, n: int, limite: float, 
                            max_combinacoes: int = 10000) -> dict:
    """
    Orquestra toda a análise de amostragem com abordagem educativa.
    
    Args:
        valores: Valores da população
        n: Tamanho da amostra
        limite: Limite para proporções
        max_combinacoes: Limite para cálculo exaustivo
        
    Returns:
        Dict com resultados completos
    """
    st.write("### �� Executando Análise de Amostragem...")
    
    progress_bar = st.progress(0)
    status_text = st.empty()
    
    try:
        # 1. VALIDAÇÃO ESTATÍSTICA
        status_text.text("Validando parâmetros estatísticos...")
        progress_bar.progress(10)
        
        validacao = validar_entradas(valores, n, len(valores), limite)
        
        if not validacao['sucesso']:
            st.error("## ❌ Erros de Validação Estatística:")
            for erro in validacao['erros']:
                st.error(f"- {erro}")
            return {'sucesso': False, 'erros': validacao['erros']}
        
        # Alertas informativos
        if validacao['avisos']:
            st.warning("## ⚠️ Considerações Estatísticas:")
            for aviso in validacao['avisos']:
                st.warning(f"- {aviso}")
        
        # 2. GERAÇÃO DE AMOSTRAS
        status_text.text("Gerando amostras...")
        progress_bar.progress(40)
        
        resultado_amostras = gerar_amostras_inteligente(valores, n, max_combinacoes)
        
        # 3. CÁLCULO DE ESTATÍSTICAS
        status_text.text("Calculando estatísticas...")
        progress_bar.progress(70)
        
        estatisticas = calcular_estatisticas(
            resultado_amostras['combinacoes'], 
            limite
        )
        
        # 4. ANÁLISE EXPLORATÓRIA
        status_text.text("Realizando análise exploratória...")
        progress_bar.progress(90)
        
        resultado_final = {
            'sucesso': True,
            'dados': estatisticas,
            'metadados': {
                'metodo': resultado_amostras['metodo'],
                'total_combinacoes': resultado_amostras['total_combinacoes'],
                'amostras_geradas': resultado_amostras['amostras_geradas'],
                'populacao': valores,
                'parametros': {'n': n, 'limite': limite},
                'timestamp': datetime.now().strftime("%Y-%m-%d %H:%M:%S")
            },
            'validacao': validacao
        }
        
        progress_bar.progress(100)
        status_text.text("✅ Análise concluída!")
        
        return resultado_final
        
    except Exception as e:
        st.error(f"## ❌ Erro na Análise: {str(e)}")
        st.code(traceback.format_exc())
        return {'sucesso': False, 'erro': str(e)}

# INTERFACE PRINCIPAL =========================================================

def main():
    st.title("�� Laboratório de Amostragem Estatística")
    st.markdown("""
    ### Material Didático para Ensino de Inferência Estatística
    
    Este laboratório virtual permite explorar conceitos fundamentais de amostragem:
    - **Distribuições amostrais**
    - **Teorema do Limite Central** 
    - **Variabilidade das estimativas**
    - **Validação estatística**
    """)
    
    # SIDEBAR - PARÂMETROS DA ANÁLISE
    with st.sidebar:
        st.header("⚙️ Configuração da Análise")
        
        # Parâmetros principais
        N = st.number_input('Tamanho da população (N):', 
                           min_value=2, max_value=100, value=8,
                           help="Número total de elementos na população")
        
        valores_input = st.text_input('Valores da população (separados por vírgula):', 
                                     '1,2,4,4,7,7,7,8',
                                     help="Digite os valores numéricos separados por vírgula")
        
        n = st.number_input('Tamanho da amostra (n):', 
                           min_value=1, max_value=20, value=4,
                           help="Tamanho de cada amostra (deve ser ≤ N)")
        
        limite = st.number_input('Valor limite para proporção:', value=4.0,
                               help="Valor limite para cálculo de proporções")
        
        max_combinacoes = st.slider('Máximo de combinações para análise:', 
                                   min_value=100, max_value=10000, 
                                   value=5000, step=100,
                                   help="Limite para análise exaustiva vs amostragem")
        
        if st.button('�� Executar Análise', type='primary', use_container_width=True):
            # Processar valores da população
            try:
                valores = [float(x.strip()) for x in valores_input.split(',')]
                
                if len(valores) != N:
                    st.warning(f"⚠️ Número de valores ({len(valores)}) diferente do N informado ({N})")
                
                # Executar análise completa
                resultado = executar_analise_completa(
                    valores, n, limite, max_combinacoes
                )
                
                st.session_state['resultado'] = resultado
                
            except ValueError as e:
                st.error("❌ Erro: Certifique-se de que todos os valores são numéricos")
            except Exception as e:
                st.error(f"❌ Erro inesperado: {str(e)}")
    
    # ÁREA PRINCIPAL - RESULTADOS E ANÁLISE
    if 'resultado' in st.session_state:
        resultado = st.session_state['resultado']
        
        if resultado['sucesso']:
            st.success(f"## ✅ Análise Concluída!")
            
            # Metadados da análise
            col1, col2, col3 = st.columns(3)
            with col1:
                st.metric("Método", resultado['metadados']['metodo'].upper())
            with col2:
                st.metric("Amostras Geradas", resultado['metadados']['amostras_geradas'])
            with col3:
                st.metric("Combinações Possíveis", resultado['metadados']['total_combinacoes'])
            
            # Tabela de resultados
            st.subheader("�� Resultados Detalhados")
            st.dataframe(resultado['dados'].head(10), use_container_width=True,
                        caption="Primeiras 10 amostras com suas estatísticas")
            
            # Análise exploratória
            gerar_analise_exploratoria(resultado['dados'])
            
            # Visualizações educativas
            st.subheader("�� Visualizações das Distribuições Amostrais")
            
            col1, col2 = st.columns(2)
            with col1:
                st.bar_chart(resultado['dados']['media'])
                st.caption("**Distribuição das Médias Amostrais** - Ilustra o Teorema do Limite Central")
            
            with col2:
                st.bar_chart(resultado['dados']['proporcao'])
                st.caption("**Distribuição das Proporções** - Mostra a variabilidade das estimativas")
            
            # Download para relatórios
            st.subheader("�� Exportar Resultados")
            csv = resultado['dados'].to_csv(index=False)
            st.download_button(
                label="Download CSV Completo",
                data=csv,
                file_name=f"resultados_amostragem_{datetime.now().strftime('%Y%m%d_%H%M%S')}.csv",
                mime="text/csv",
                help="Baixe os resultados completos para análise offline"
            )
        
        else:
            st.error("## ❌ Análise Falhou")
            if 'erros' in resultado:
                for erro in resultado['erros']:
                    st.error(f"- {erro}")

if __name__ == "__main__":
    main()
\end{lstlisting}

\section{Conclusão e Aplicações Didáticas}
\label{sec:conclusao}

\subsection{Síntese dos Conceitos Abordados}

As implementações apresentadas incorporam os seguintes conceitos fundamentais de inferência estatística:

\begin{itemize}
\item \textbf{Distribuição Amostral}: Comportamento das estatísticas sobre todas as amostras possíveis
\item \textbf{Teorema do Limite Central}: Fundamentação teórica para a normalidade das médias amostrais
\item \textbf{Validação Estatística}: Verificação de pressupostos e condições para inferência
\item \textbf{Eficiência Computacional}: Balanceamento entre exatidão e viabilidade prática
\item \textbf{Visualização Educativa}: Representações gráficas para compreensão conceitual
\end{itemize}

\subsection{Aplicações em Contexto Educacional}

Estas implementações podem ser utilizadas para:

\begin{enumerate}
\item \textbf{Aulas de Estatística Básica}: Demonstração prática de conceitos teóricos
\item \textbf{Laboratórios de Inferência}: Experimentação com diferentes parâmetros populacionais
\item \textbf{Validação de Pressupostos}: Verificação empírica de teoremas estatísticos
\item \textbf{Desenvolvimento de Competências}: Integração de programação e estatística
\end{enumerate}

\subsection{Extensões e Adaptações}

Para adaptar o material a diferentes contextos educacionais:

\begin{itemize}
\item \textbf{Nível Básico}: Focar nas visualizações e interpretação intuitiva
\item \textbf{Nível Intermediário}: Explorar diferentes distribuições populacionais
\item \textbf{Nível Avançado}: Implementar intervalos de confiança e testes de hipóteses
\item \textbf{Pesquisa}: Estender para amostragem complexa e bootstrap
\end{itemize}

\section{Anotações das Aulas}
\label{sec:anotacoes-aulas}

\subsection{Aula 03 - Fundamentos de Amostragem Estatística}
\label{subsec:aula03}

As anotações da Aula 03 abordam conceitos fundamentais de amostragem estatística, incluindo distribuições amostrais, probabilidades de inclusão e estimadores. O material complementa os códigos apresentados nas seções anteriores, fornecendo a base teórica para as implementações práticas.

\subsubsection{Principais Tópicos Abordados}
\begin{itemize}
\item \textbf{Probabilidades de Inclusão}: Conceitos de probabilidades de primeira e segunda ordem (\(\pi_k\) e \(\pi_{kl}\))
\item \textbf{Estimadores}: Fundamentos dos estimadores de Horvitz-Thompson e suas propriedades
\item \textbf{Variância Amostral}: Cálculos e propriedades da variância em amostragem
\item \textbf{Distribuições Amostrais}: Comportamento das estatísticas amostrais e teorema do limite central
\item \textbf{Planejamento Amostral}: Estratégias para seleção de amostras representativas
\end{itemize}

\subsubsection{Inclusão do Material Original}
\includepdf[pages=-, scale=0.75, pagecommand={\thispagestyle{plain}}]{Anotacoes_da_aula_03.pdf}

\subsection{Relação com as Implementações Práticas}
\label{subsec:contextualizacao}

As anotações teóricas da Aula 03 se relacionam diretamente com as implementações práticas apresentadas neste documento:

\begin{table}[h]
\centering
\begin{tabular}{p{0.3\linewidth} p{0.6\linewidth}}
\toprule
\textbf{Conceito Teórico} & \textbf{Implementação Prática} \\
\midrule
Probabilidades de Inclusão & Validação de entradas e verificação de consistência dos parâmetros \\
Distribuições Amostrais & Geração de todas as amostras possíveis ou amostragem representativa \\
Estimadores & Cálculo de médias, proporções e outras estatísticas descritivas \\
Variância Amostral & Cálculo do desvio padrão e análise da precisão das estimativas \\
Teorema do Limite Central & Análise da distribuição das médias amostrais \\
\bottomrule
\end{tabular}
\caption{Relação entre teoria e prática nas implementações}
\end{table}

\subsection{Exercícios Complementares Baseados na Aula}
\label{subsec:exercicios-aula}

Com base nas anotações da Aula 03, os seguintes exercícios complementares são sugeridos para aprofundamento:

\subsubsection{Exercício 1: Probabilidades de Inclusão}
Calcular as probabilidades de inclusão de primeira e segunda ordem para diferentes tamanhos amostrais e analisar como essas probabilidades afetam a precisão das estimativas.

\subsubsection{Exercício 2: Comparação de Estimadores}
Implementar e comparar diferentes estimadores (Horvitz-Thompson, razão, regressão) para os mesmos dados populacionais, analisando viés e eficiência.

\subsubsection{Exercício 3: Análise de Variância}
Comparar a variância teórica prevista pelas fórmulas com a variância observada nas simulações de Monte Carlo.

\subsubsection{Exercício 4: Tamanho Amostral Ótimo}
Desenvolver um algoritmo para determinar o tamanho amostral que minimiza o erro amostral para diferentes orçamentos e precisões requeridas.

\subsubsection{Exercício 5: Efeito do Plano Amostral}
Analisar como diferentes planos amostrais (aleatório simples, estratificado, por conglomerados) afetam a variância das estimativas.

\begin{lstlisting}[style=r-style, caption={Exemplo de cálculo de probabilidades de inclusão}]
#' Cálculo de Probabilidades de Inclusão
#' Baseado nos conceitos da Aula 03

calcular_probabilidades_inclusao <- function(N, n) {
  # Probabilidade de inclusão de primeira ordem
  pi_k <- n/N
  
  # Probabilidade de inclusão de segunda ordem (k != l)
  pi_kl <- (n * (n - 1)) / (N * (N - 1))
  
  # Matriz de probabilidades conjuntas
  matriz_pi <- matrix(pi_kl, nrow = N, ncol = N)
  diag(matriz_pi) <- pi_k
  
  return(list(
    primeira_ordem = pi_k,
    segunda_ordem = pi_kl,
    matriz = matriz_pi
  ))
}

# Exemplo para população de tamanho 8 e amostra de tamanho 4
probabilidades <- calcular_probabilidades_inclusao(8, 4)

cat("Probabilidade de primeira ordem:", probabilidades$primeira_ordem, "\n")
cat("Probabilidade de segunda ordem:", probabilidades$segunda_ordem, "\n")
\end{lstlisting}


\end{document}