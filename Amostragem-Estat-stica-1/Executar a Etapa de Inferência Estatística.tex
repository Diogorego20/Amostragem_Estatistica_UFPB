Como Executar a Etapa de Inferência Estatística

1. O Que é Inferência Estatística e Qual Seu Papel no Projeto?

Tentando exemplar em planejar o levantamento, estruturar a coleta e processar os dados. A inferência estatística é a etapa final e culminante desse processo: é aqui que você usará os dados da sua amostra para fazer afirmações, tirar conclusões e estimar características sobre a população inteira de agricultores familiares da Paraíba.

Você já construiu as ferramentas para chegar aos dados (ETL) e para planejar a amostra (aplicativo Shiny). Agora, o objetivo é transformar os dados brutos em conhecimento acionável. A inferência é o que permite que você responda às perguntas que motivaram o levantamento em primeiro lugar.


A inferência estatística nos dá a capacidade de generalizar os achados de uma amostra para a população mais ampla, quantificando a incerteza inerente a esse processo por meio de probabilidades. [1]

2. Da Preparação à Análise: Passos Práticos

Com o seu frame_integrado.csv em mãos (o resultado do seu pipeline de ETL), o caminho para a inferência está pronto. O processo se divide em duas grandes áreas: estimação e testes de hipóteses.

Tipo de Análise
Objetivo
Exemplo no seu Projeto
Estimação
Estimar um valor desconhecido da população (como uma média ou proporção).
"Qual a renda média mensal dos agricultores familiares da Paraíba?" ou "Qual a proporção de agricultores que utilizam sementes crioulas?"
Teste de Hipóteses
Verificar se uma afirmação (hipótese) sobre a população é estatisticamente válida.
"A proporção de agricultores com acesso a crédito no Sertão é diferente da proporção no Agreste?"


3. Como Realizar a Análise Inferencial

Vamos simular um cenário onde seu frame_integrado.csv já foi gerado. Ele conteria colunas como municipio_ibge, renda_familiar_mensal, usa_credito (sim/não), area_plantada_ha, etc.

Passo 1: Análise Descritiva (Pré-requisito)

Antes da inferência, você precisa entender sua amostra. Calcule estatísticas descritivas básicas para as variáveis de interesse. Isso ajuda a identificar padrões e a formular hipóteses.

•
Para variáveis numéricas (renda, área plantada): média, mediana, desvio padrão, mínimo, máximo.

•
Para variáveis categóricas (usa_credito, regiao): tabelas de frequência e proporções.

Passo 2: Estimação por Intervalo de Confiança

Uma estimativa pontual (como a média da amostra) é útil, mas incompleta. O intervalo de confiança (IC) é a ferramenta profissional correta, pois ele fornece uma faixa de valores onde o verdadeiro parâmetro da população provavelmente está, com um certo nível de confiança (geralmente 95%).

Exemplo Prático: Estimar a proporção de agricultores com acesso a crédito.

1.
Calcular a proporção na amostra (p̂): Suponha que na sua amostra de 500 agricultores, 150 responderam "sim" para usa_credito. Então, p̂ = 150 / 500 = 0.30.

2.
Calcular o erro padrão e o IC: Usando as fórmulas que você já implementou no seu app Shiny, você calcularia o intervalo de confiança de 95% para essa proporção.

3.
Interpretar o resultado: O resultado poderia ser "A proporção de agricultores com acesso a crédito na amostra é de 30%. Com 95% de confiança, estimamos que a verdadeira proporção na população de agricultores familiares da Paraíba está entre 26% e 34%."

Passo 3: Testes de Hipóteses

Os testes de hipóteses são usados para tomar decisões baseadas em dados.

Exemplo Prático: Testar se a renda média no Cariri é igual à do Litoral.

1.
Formular as hipóteses:

•
Hipótese Nula (H₀): A renda média no Cariri é igual à renda média no Litoral.

•
Hipótese Alternativa (H₁): A renda média no Cariri é diferente da renda média no Litoral.



2.
Escolher o teste estatístico: Como você está comparando as médias de dois grupos independentes, um Teste T de duas amostras seria apropriado.

3.
Calcular a estatística de teste e o p-valor: O software estatístico (R ou Python) fará isso por você.

4.
Concluir: Se o p-valor for menor que 0.05 (o nível de significância padrão), você rejeita a hipótese nula. A conclusão seria: "Há evidências estatísticas para afirmar que a renda média dos agricultores familiares difere significativamente entre as regiões do Cariri e do Litoral (p < 0.05)."

4. Como Adicionar ao Seu Trabalho

Crie uma nova seção no seu relatório principal ou um novo documento de análise.

Seção Sugerida: 11. Análise Inferencial dos Dados


"Após a integração e limpeza dos dados, procedeu-se à análise inferencial para responder aos objetivos do levantamento. Foram utilizadas técnicas de estimação por intervalo de confiança e testes de hipóteses para generalizar os resultados da amostra para a população de agricultores familiares da Paraíba.11.1. Perfil de Acesso a Crédito Rural A análise da amostra revelou que 30% dos agricultores fazem uso de crédito rural. Utilizando um estimador para populações finitas e considerando o desenho amostral, o intervalo de confiança de 95% para esta proporção na população foi calculado entre [0.26, 0.34]. Isso indica que, com alta confiança, a verdadeira proporção de agricultores com acesso a crédito no estado está entre 26% e 34%.11.2. Comparativo de Renda entre Regiões Para avaliar se existem disparidades de renda entre as mesorregiões, foi realizado um teste de hipóteses. A hipótese nula de que as rendas médias são iguais foi testada contra a alternativa de que são diferentes. O resultado do Teste T (t = X.XX, p = 0.02) indicou um p-valor significativo, levando à rejeição da hipótese nula. Conclui-se que existem diferenças estatisticamente significantes na renda média dos agricultores entre as regiões A e B..."

Na próxima etapa, forneceremos exemplos de código em R e Python para implementar essas análises.




Referências [1]: # "Bussab, W. O., & Morettin, P. A. (2017). Estatística Básica. 9ª ed. São Paulo: Saraiva Educação."

