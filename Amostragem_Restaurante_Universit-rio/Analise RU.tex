
% !TeX program = pdflatex
% Modelo ABNT/abnTeX2 para relatório técnico-científico
\documentclass[
   12pt,                 % tamanho da fonte
   oneside,              % impressão em apenas um lado
   a4paper,              % papel A4
   brazil                % idioma principal
]{abntex2}

% ------------------------------------------------------
% Pacotes essenciais
% ------------------------------------------------------
\usepackage{lmodern}            % fonte Latin Modern
\usepackage[T1]{fontenc}        % codificação da fonte
\usepackage[utf8]{inputenc}     % utf-8
\usepackage{indentfirst}        % indenta primeiro parágrafo
\usepackage{graphicx}           % figuras
\usepackage{microtype}          % melhorias tipográficas
\usepackage{amsmath,amssymb}    % matemática
\usepackage{booktabs}           % tabelas profissionais
\usepackage{multirow}           % tabelas
\usepackage{siunitx}            % valores numéricos
\usepackage{caption}            % legendas
\usepackage{subcaption}         % subfiguras
\usepackage{hyperref}           % links
\usepackage{enumitem}           % listas customizadas
\usepackage{longtable}          % tabelas longas
\usepackage{float}              % controle de float
\usepackage{setspace}           % espaçamento
\usepackage{pdfpages}           % inserir PDFs
\usepackage{url}                % URLs
\usepackage{ragged2e}           % alinhamento
\usepackage{geometry}           % margens

% Margens padrão ABNT
\geometry{
  left=3cm,
  right=2cm,
  top=3cm,
  bottom=2cm
}

% ------------------------------------------------------
% Configuração de idioma e citações ABNT
% ------------------------------------------------------
\usepackage[brazil]{babel}
\usepackage[alf]{abntex2cite} % estilo autor-data ABNT (alf). Para numérico, use [num]

% ------------------------------------------------------
% Metadados do documento
% ------------------------------------------------------
\titulo{Impacto da Redução de Preço no Restaurante Universitário da UFPB sobre a Qualidade Percebida}
\autor{Diogo Rego}
\instituicao{%
Universidade Federal da Paraíba (UFPB)\\
Centro de Ciências Exatas e da Natureza (CCEN)\\
Departamento de Estatística
}
\local{João Pessoa -- PB}
\data{\today}

% ------------------------------------------------------
% Configurações de PDF
% ------------------------------------------------------
\hypersetup{
  pdftitle={Impacto da Redução de Preço no RU/UFPB sobre a Qualidade Percebida},
  pdfauthor={Diogo Rego},
  pdfsubject={Relatório técnico-científico},
  pdfcreator={LaTeX com abnTeX2},
  pdfkeywords={RU, UFPB, qualidade percebida, amostragem, Likert, regressão, qui-quadrado},
  colorlinks=true,
  linkcolor=blue,
  citecolor=blue,
  urlcolor=blue
}

% ------------------------------------------------------
% Início do documento
% ------------------------------------------------------
\begin{document}
\selectlanguage{brazil}
\frenchspacing             % retira espaço extra entre frases

% Capa
\imprimircapa

% Folha de rosto
\imprimirfolhaderosto*

% Dedicatória (opcional)
% \begin{dedicatoria}
%     \vspace*{\fill}
%     \centering
%     \textit{Dedico este trabalho a ...}
%     \vspace*{\fill}
% \end{dedicatoria}

% Agradecimentos (opcional)
% \begin{agradecimentos}
%     Texto dos agradecimentos.
% \end{agradecimentos}

% Resumo
\begin{resumo}
Este relatório investiga se a redução recente no preço do Restaurante Universitário (RU) da UFPB impactou a qualidade percebida dos alimentos pelos estudantes. Aplicou-se questionário com escala Likert (1--5) cobrindo dimensões como sabor, variedade, frescor, temperatura e higiene, além de questões sobre percepção antes e após a redução de preço.

O plano amostral considerou estudantes usuários do RU, por amostragem aleatória simples ou estratificada por curso/turno. A análise compreende testes de associação (qui-quadrado), comparação de médias (ANOVA) e modelos de regressão para controle de confundidores (frequência de uso e perfil socioeconômico). Resultados e implicações práticas para gestão do RU são discutidos.

\textbf{Palavras-chave}: Restaurante Universitário; Qualidade Percebida; Escala Likert; Amostragem; Regressão; Qui-quadrado.
\end{resumo}

% Lista de figuras e tabelas (opcionais)
\listoffigures
\listoftables

% Sumário
\tableofcontents

% ------------------------------------------------------
% Corpo do texto
% ------------------------------------------------------
\chapter{Introdução}
\label{chap:introducao}
\justifying
Em \textit{data recente}, a UFPB adotou novo modelo de subsídio ao RU, com desconto significativo para estudantes de graduação e pós-graduação. Este trabalho busca inferir se tal redução de preço influenciou a \textit{qualidade percebida} das refeições oferecidas.

A pergunta central é: \textbf{a redução no preço impactou a qualidade percebida dos alimentos?} Para tal, mensuramos dimensões de qualidade (sabor, variedade, frescor, temperatura e higiene), além de expectativas e \textit{antes vs. depois} da política de preço.

% Exemplo de figura com a imagem da notícia (substitua o caminho)
\begin{figure}[H]
    \centering
    \includegraphics[width=\textwidth]{}
\begin{figure}
        \centering
        \includegraphics[width=0.7\linewidth]{}
\begin{figure}
            \centering
            \includegraphics[width=0.7\linewidth]{}
\begin{figure}
                \centering
                \includegraphics[width=0.9\linewidth]{ru1.jpeg}
                \label{fig:placeholder}
            \end{figure}
            \label{fig:placeholder}
        \end{figure}
                \caption{https://g1.globo.com/pb/paraiba/noticia/2025/11/11/estudantes-de-graduacao-e-pos-passam-a-ter-desconto-de-60percent-no-restaurante-universitario-da-ufpb.ghtml}
        \label{fig:placeholder}
    \end{figure}
    \label{fig:contexto_ru}
\end{figure}

\chapter{Objetivo e Hipóteses}
\label{chap:objetivo}
\section{Objetivo Geral}
Inferir se a redução no preço do RU influencia a qualidade percebida dos alimentos.

\section{Hipóteses}
\begin{description}[leftmargin=1cm]
  \item[H\textsubscript{0}] A redução de preço \textbf{não} influencia a qualidade percebida.
  \item[H\textsubscript{1}] A redução de preço \textbf{influencia} a qualidade percebida.
\end{description}

\chapter{Variáveis e Instrumento}
\label{chap:variaveis}
\section{Variáveis}
\begin{itemize}
    \item \textbf{Independente}: indicador da redução de preço (\textit{antes} vs. \textit{depois}) ou fator binário (0/1).
    \item \textbf{Dependente}: qualidade percebida (Likert 1--5) nas dimensões: sabor, variedade, frescor, temperatura, higiene.
    \item \textbf{Controle}: frequência de uso do RU, perfil socioeconômico, expectativas anteriores, curso/turno.
\end{itemize}

\section{Estrutura do Questionário}
\subsection*{Seção A -- Perfil do Respondente}
Curso e nível (graduação/pós), frequência no RU (diária, semanal, ocasional), turno.

\subsection*{Seção B -- Percepção da Qualidade (Likert 1--5)}
Avaliação de \textit{sabor, variedade, temperatura, higiene}.

\subsection*{Seção C -- Expectativas e Mudança}
``A qualidade mudou após a redução do preço?'' (melhorou/manteve/piorou) e comparação \textit{antes vs. depois}.

\chapter{Plano Amostral}

\begin{figure} [H]
            \centering
            \includegraphics[width=0.7\linewidth]{}
\begin{figure}
                \centering
                \includegraphics[width=0.7\linewidth]{}
\begin{figure} 
                    \centering
                    \includegraphics[width=0.8\linewidth]{pre;o.jpeg}
                                       \label{fig:placeholder}
                \end{figure}
                                \label{fig:placeholder}
            \end{figure}
            \label{fig:placeholder}
        \end{figure}

\label{chap:amostragem}
\section{População e Amostragem}
População-alvo: estudantes que utilizam o RU contemplados com os auxílio na condicao de integral, parcial a ser pago pelo acesso de acordo com a reportagem. Estratégias de amostragem:
\begin{enumerate}[label=\alph*)]
    \item \textbf{Aleatória simples}: seleção uniforme dentre usuários, envio de formulário de perguntas via e-mail dos estudantes contemplados com os auxílios integral, parcial.
    \item \textbf{Estratificada}: por curso, centro, turno, ou frequência de uso, para garantir representatividade.
\end{enumerate}

\section{Tamanho da Amostra}
Para estimar proporção (p.ex., proporção de estudantes que percebem melhora), pode-se usar:
\begin{equation}
    n \;=\; \frac{Z^2 \cdot p \cdot (1-p)}{e^2},
    \label{eq:tam_amostra}
\end{equation}
onde $Z$ é o quantil da Normal padrão (\textit{ex.}, $Z=1{,}96$ para 95\%), $p$ a proporção esperada (\textit{ex.}, $p=0{,}5$ quando desconhecida), e $e$ o erro amostral desejado (\textit{ex.}, $e=0{,}05$).

Quando o tamanho da população $N$ é conhecido e não muito grande, aplica-se a correção para população finita:
\begin{equation}
    n_{aj} \;=\; \frac{n}{1 + \frac{n-1}{N}}.
\end{equation}

\chapter{Procedimentos de Coleta e Ética}
\label{chap:coleta}
\begin{itemize}
    \item Aplicação do formulário via plataforma online, via e-mail, ou cartazes espalhados na entrada do restaurante. 



\begin{figure}[H]
    \centering
    \includegraphics[width=0.7\linewidth]{form.png}
    \caption{Formulário interativo disponível em}
    
\end{figure}
    \item Consentimento livre e esclarecido; anonimização dos dados; aprovação em comitê, se aplicável.
\begin{figure}[H]
    \centering
    \includegraphics[width=0.7\linewidth]{LPD.jpeg}
    \caption{Respeitando a LPD}
    
\end{figure}
    


\chapter{Plano de Análise Estatística}
\label{chap:analise}
\section{Preparação}
\begin{itemize}
    \item Limpeza de dados (valores faltantes, consistência de respostas).
    \item Construção de escores de qualidade (média/mediana das dimensões).
\end{itemize}

\section{Testes}
\begin{enumerate}[label=\alph*)]
    \item \textbf{Qui-quadrado de independência}: associação entre ``mudança percebida'' (melhorou/manteve/piorou) e período (antes/depois).
    \item \textbf{ANOVA / t-test}: comparação de médias de escores de qualidade \textit{antes vs. depois}. Para Likert agregado, considere ANOVA robusta ou testes não-paramétricos (Mann-Whitney).
    \item \textbf{Regressão}: modelos lineares ou ordenados (logit/probit ordenado) com controles: frequência de uso, perfil socioeconômico, curso/turno.
\end{enumerate}

\section{Relato dos Resultados}
Apresente estimativas pontuais, intervalos de confiança, valores-$p$ e tamanhos de efeito (Cohen's $d$, $\eta^2$). Discuta relevância prática, não apenas significância estatística.

\chapter{Resultados (Modelo de Apresentação)}
\label{chap:resultados}
\section{Descritivos}
\begin{table}[H]
\centering
\caption{Distribuição das respostas por dimensão de qualidade (Likert 1--5).}
\label{tab:likert}
\begin{tabular}{lcccccc}
\toprule
Dimensão & 1 & 2 & 3 & 4 & 5 & Média \\
\midrule
Sabor        & 5\% & 10\% & 25\% & 35\% & 25\% & 3{,}65 \\
Variedade    & 8\% & 15\% & 30\% & 30\% & 17\% & 3{,}33 \\
Temperatura  & 6\% & 12\% & 28\% & 32\% & 22\% & 3{,}52 \\
Higiene      & 4\% & 9\%  & 26\% & 34\% & 27\% & 3{,}71 \\
\bottomrule
\end{tabular}
\end{table}

\section{Qui-quadrado}
\begin{table}[H]
\centering
\caption{Cruzamento: mudança percebida vs. período (antes/depois).}
\label{tab:chi}
\begin{tabular}{lccc}
\toprule
                & Melhorou & Manteve & Piorou \\
\midrule
Antes           & 80       & 120     & 50     \\
Depois          & 150      & 100     & 30     \\
\bottomrule
\end{tabular}

\vspace{0.5em}
Estatística $\chi^2 = 12{,}45$, $gl=2$, $p=0{,}002$ (exemplo). % Substitua pelos seus resultados
\end{table}

\section{ANOVA / Regressão}
Inclua aqui tabelas de resultados (coeficientes, ICs, $p$) e gráficos de efeitos.

\chapter{Discussão}
\label{chap:discussao}
Interprete os achados, discuta potenciais vieses (ex.: seleção, memória), limitações da escala Likert e considerações sobre causalidade. Contextualize com a mudança de subsídio no RU.

\chapter{Conclusões e Recomendações}
\label{chap:conclusoes}
Resumo das evidências, implicações para gestão do RU (cardápio, logística, comunicação) e sugestões para monitoramento contínuo.

\chapter{Referências}
\bibliography{referencias}

\appendix
\chapter{Instrumento de Coleta (Questionário)}`


Inclua versão final do questionário (PDF ou tabelado). Pode-se anexar um PDF com \texttt{\\includepdf}.

\chapter{Código de Análise (R/Python)}
Liste aqui os scripts principais (ou inclua como arquivo externo).

% Exemplo de inclusão de código simples (texto)
\section*{Exemplo de cálculo de tamanho de amostra}
O cálculo da Equação~\ref{eq:tam_amostra} pode ser reproduzido em R/Python:
\begin{verbatim}
# R
Z <- 1.96; p <- 0.5; e <- 0.05
n <- (Z^2 * p * (1 - p)) / (e^2)

# Python
Z = 1.96; p = 0.5; e = 0.05
n = (Z**2 * p * (1 - p)) / (e**2)
\end{verbatim}

\end{document}
